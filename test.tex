\documentclass[12pt]{article}
\usepackage{graphicx} % Required for inserting images
\usepackage{amsmath}
\usepackage{amssymb}
\usepackage{amsthm}
\usepackage{xfrac}
\usepackage{mathtools}
\usepackage{relsize}
\usepackage{tikz}
\usepackage{tikz-cd} 
\usepackage{halloweenmath}
\usetikzlibrary{shapes.geometric}

\makeatletter
\renewcommand*\env@matrix[1][*\c@MaxMatrixCols c]{%
  \hskip -\arraycolsep
  \let\@ifnextchar\new@ifnextchar
  \array{#1}}
\makeatother

\newcommand{\ops}{\mathcal{L}}
\newcommand{\reals}{\mathbb{R}}
\newcommand{\q}{\mathbb{Q}}
\newcommand{\nats}{\mathbb{N}}
\newcommand{\ints}{\mathbb{Z}}
\newcommand{\tr}{\text{tr}}
\newcommand{\spann}{\text{span}}
\newcommand{\complex}{\mathbb{C}}
\newcommand{\iprod}[2]{\langle #1, #2 \rangle}
\newcommand{\proj}[2]{\text{proj}_{#1}(#2)}
\newcommand{\makelarge}[1]{\mathlarger{\mathlarger{\mathlarger{#1}}}}
\newcommand{\interior}{\text{int}}
\newcommand{\boundary}{\text{bdry}}
\newcommand{\cover}{\mathcal{O}}
\newcommand{\g}[1]{\langle #1 \rangle}
\newcommand{\ba}{\mathfrak{B}}
\newcommand{\norm}{\text{Norm}}
\newcommand{\ord}{\text{ord}}
\newcommand{\im}{\text{im}}
\newcommand{\lcm}{\text{lcm}}
\newcommand{\id}{\text{id}}
\newcommand{\pf}[2]{\dfrac{\partial #1}{\partial #2}}
\newcommand{\Aut}{\text{Aut}}
\newcommand{\Alt}{\text{Alt}}
\newcommand{\sgn}{\text{sgn}}
\newcommand{\grad}{\text{grad}}
\newcommand{\curl}{\text{curl}}
\newcommand{\divv}{\text{div}}

\title{Tutorial 8 solutions - MAT267}
\author{Isaac Clark}
\date{March 14th, 2025}

\begin{document}

\maketitle 

\noindent Consider the second-order ODE $$\alpha(x) u'' + p(x) u' + q(x)u = r(x) \ , \hspace{1.5cm} x \in [a,b] \subseteq \reals$$ where the coefficient functions $\alpha(x), p(x), q(x), r(x)$ are assumed to be continuous real-valued functions on $I = [a, b]$. If $\alpha(x) \equiv 1$, then the above equation is said to be in normal form, so we get $$u'' + p(x) u' + q(x) u = r(x) \ , \hspace{1.5cm} x \in [a, b]$$ Provided $\alpha(x) \not= 0$ on $[a, b]$, we may divide by $\alpha(x)$ to obtain an equivalent ODE in normal form. We will mostly focus on the homogeneous case: $$u'' + p(x) u' + q(x) u = 0 \ , \hspace{1.5cm} x \in [a, b]$$ We can then transform the above into a system by introducing $w = u'$ and apply the existence and uniqueness theorem to this system to obtain the existence of a unique solution to the above, given a initial condition. \\

\noindent \textbf{Definition 1}. The Wronskian of two differentiable functions $f(x), g(x)$ is: $$W(f, g, x) = f(x)g'(x) - f'(x)g(x) = \det \begin{pmatrix} f(x) & g(x) \\ f'(x) & g'(x) \end{pmatrix}$$ Clearly, if $f, g$ are linearly dependent differentiable functions, the Wronskian vanishes identically. \\

\noindent \textbf{Theorem 2}. If $f, g$ are linearly independent solutions to the homogeneous ODE, the Wronskian never vanishes. \\

\noindent \textit{Note}: we are not given nor required to know a proof for this. \newpage

\noindent \textbf{Theorem 3}. If $f, g$ are linearly independent solutions of the homogeneous ODE, then $f(x)$ must vanish at one point between any two successive zeros of $g(x)$. In other words, the zeroes of $f(x)$ and $g(x)$ occur alternately. \\

\noindent \textit{Proof}: Suppose $x_1 < x_2$ are successive zeroes of $g$. Then, we may compute: $$W(f, g, x_1) = f(x_1)g'(x_1)$$ $$W(f, g, x_2) = f(x_2)g'(x_2)$$ Further, $g'(x_1) \cdot g'(x_2) \leq 0$, since if not, then $g'(x_1), g'(x_2)$ would both be non-zero with the same sign, so, for example, $g$ would "pass through" each zero from below, but, by the intermediate value theorem, $g$ would then have to obtain a zero in between $x_1, x_2$, which contradicts our assumption that they are successive zeros of $g(x)$. Now assume, for a contradiction, that $f \not= 0$ on $[x_1, x_2]$, then, again by the intermediate value theorem, $f < 0$ or $f > 0$ on $[x_1, x_2]$. In either case we have $W(f, g, x_1) \cdot W(f, g, x_2) \leq 0$. If $W(f, g, x_1) \cdot W(f, g, x_2) = 0$ then one of $W(f, g, x_1), W(f, g, x_2)$ must be zero, which would contradict that $f(x), g(x)$ are linearly independent. If $W(f, g, x_1) \cdot W(f, g, x_2)< 0$ then one of $W(f,g, x_1), W(f, g, x_2)$ is positive and the other negative, but, by definition, the Wronskian is simply a polynomial in $f, f', g, g'$ all continuous functions, the map $x \mapsto W(f, g, x)$ is continuous, so again by the intermediate value theorem, there is some $x^* \in (x_1, x_2)$ such that $W(f, g, x^*) = 0$, but this would again contradict that $f(x), g(x)$ are linearly independent. Thus, there must be at least one point, $x_f \in (x_1, x_2)$ such that $f(x_f) = 0$, as desired. $\qed$ \\

\noindent \textbf{Theorem 4} (Sturn comparison theorem). Let $f(x), g(x)$ be non-trivial solutions of the ODES $$u'' + p(x)u = 0$$ $$v'' + q(x)v = 0$$ respectively, where $p(x) \geq q(x)$. Then $f(x)$ vanishes at least once between any two zeroes of $g(x)$, unless $p \equiv q$ and $f$ is a constant multiple of $g$. \\

\noindent \textit{Proof}: Let $x_1 < x_2$ be two successive zeroes of $g$. Assume, for a contradiction, that $f(x) \not= 0$ for all $x \in [x_1, x_2]$. Without loss of generality (by replacing $f$ with $-f$ and/or $g$ with $-g$ as necessary) we may assume that $f > 0$ and $g > 0$ on $(x_1, x_2)$. We may compute, under these assumptions, that: $$W(f, g, x_1) = f(x_1)g'(x_1) \geq 0$$ $$W(f, g, x_2) = f(x_2) g'(x_2) \leq 0$$ Next, \begin{align*}
    \dfrac{d}{dx} W(f, g, x) &= \dfrac{d}{dx} \Big(f(x)g'(x) - f'(x)g(x) \Big) \\
    &= f(x)g''(x) - f''(x)g(x) \\
    &= f(x) \big(-q(x) g(x) \big) - \big(p(x) f(x) \big) g(x) \\
    &= f(x) g(x) \Big( p(x) - q(x) \Big)
\end{align*} Since $p(x) \geq q(x)$ and $f, g > 0$ all by assumption, we have $\dfrac{d}{dx}W(f, g, x) \geq 0$ and therefore that the map $x \mapsto W(f, g, x)$ is non-decreasing. In particular, $x_1 < x_2 \implies W(f, g, x_1) \leq W(f, g, x_2)$. But, as remarked above, $W(f, g, x_2) \leq 0 \leq W(f, g, x_1)$, so taken together $W(f, g, x_1) = W(f, g, x_2) = 0$. Again using that $W(f, g, x)$ is non-decreasing in $x$, $W(f, g, x) \equiv 0$ on $[x_1, x_2]$, and so, $f, g$ are linearly dependent by \textbf{Theorem 2}. Further, $\dfrac{d}{dx} W(f, g, x) = 0$ on $[x_1, x_2]$. Since we assumed $f, g > 0$ and thus that $f(x)g(x) > 0$ on $(x_1, x_2)$, it also follows that $p \equiv q$ on $(x_1, x_2)$, which finishes the proof. $\qed$ \\

\noindent \textbf{Application 1}. Suppose that $q(x) \leq 0$. Then no non-trivial solution of $u'' + q(x)u = 0$ can have more than one zero. \\

\noindent \textit{Proof}: First, $v \equiv 1$ is a solution to $v'' = 0$. So, if $u$ is a solution to $u'' + q(x)u = 0$ with $x_1 < x_2$ zeroes, then by \textbf{Theorem 4}, $v$ must vanish at some $x^* \in (x_1, x_2)$, which, since $v \equiv 1$, is clearly a contradiction. Thus, $u$ has at most one zero. $\qed$ \\

\noindent \textbf{Application 2}: Suppose $q(x) \geq k^2 > 0$. Then any solution of $u'' + q(x)u = 0$ must vanish between any two successive zeroes of any given function $A \cos \big( k(x - x_1) \big)$ and hence in any interval of length $\pi/k$. \\

\noindent \textit{Proof}: First, $v(x) = A \cos \big( k (x - x_1) \big)$ is a solution to $v'' + k^2v = 0$. So, if $u$ is a solution to $u'' + q(x)u = 0$, then $u$ must vanish in $(x_1, x_1 + \pi/k)$ by \textbf{Theorem 4}. Allowing $x_1$ to vary in $\reals$ then yields that $u$ must also vanish on any interval of length $\pi/k$, as desired. $\qed$ \\ \\

\noindent Recall from Tutorial 7 that the Bessel differential equation of order $n$ is: $$u'' + \dfrac{1}{x}u' + \left(1 - \dfrac{n^2}{x^2} \right)u = 0$$ whose solution is the so-called Bessel function of order $n$. We may then perform the substitution $w = u\sqrt{x}$ to obtain the equivalent ODE: $$w'' + \left( 1 - \dfrac{4n^2 - 1}{4x^2} \right)w = 0$$ whose solutions vanish whenever $u$ does, for $x \not= 0$, as is apparent from our substitution. \\

\noindent \textbf{Application 3}: Every interval of length $\pi$ of the positive $x$-axis contains at least $1$ zero of any solution of the Bessel ODE of order zero, and at most $1$ zero of any non-trivial solution of the Bessel ODE of order $n > 1/2$. \\

\noindent \textit{Proof}: If $n = 0$, then $q(x) = 1- (4n^2 - 1)/4x^2 \equiv 1 \geq 1^2 > 0$, so, by \textbf{Application 2}, if $u$ solves $u'' + q(x) u = 0$, i.e. the Bessel ODE of order $0$, then it must vanish on every interval of length $\pi$. If $n > 1/2$, then $q(x) = 1 - (4n^2 - 1)/4x^2 \leq 0$, so, by \textbf{Application 1}, any solution to $u'' + q(x)u = 0$, i.e. any non-trivial solution to the Bessel ODE of order $n$, can have at most one zero. $\qed$

\end{document}
